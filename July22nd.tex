\documentclass{ltjsarticle}


% 数式
\usepackage[]{amsmath, amssymb, amsfonts, amsthm}
\usepackage{physics}
\usepackage{enumerate}

% 数式以外
% \usepackage[]{comment}
\usepackage[longnamesfirst]{natbib}
\usepackage[]{graphicx}
% \usepackage[]{here}


% 定理や定義の書式

% 定義
\theoremstyle{definition}
\newtheorem{dfn}{Definition}[subsubsection]
% 定理
\newtheorem{thm}[dfn]{Theorem}
\newtheorem{thm*}[dfn]{Theorem*}
% 命題
\newtheorem{prop}[dfn]{Proposition}
% 補題
\newtheorem{lem}[dfn]{Lemma}
% 注意
\newtheorem{rem}[dfn]{Rem}
% アルゴリズム
\newtheorem{algo}[dfn]{Algorithm}
% 例
\newtheorem{ex}[dfn]{Example}

% Declear new operator
\DeclareMathOperator*{\plim}{plim}



\begin{document}

\title{確率解析勉強会発表資料}
\author{中津陽}
\maketitle

$I(\Delta)^2$は条件付き期待値のJensenの不等式により、非負のsubmartingaleである。
したがって、Karatzas, and Shreveより、$I(\Delta)$は一意なDoob-Meyer decompositionを持つ。
\begin{equation}
    I_t(\Delta) = M_t + A_t, \, 0 \leq t < \infty
\end{equation}
ただし、$M$は右連続なmartingaleで、$A$は連続な増加過程である。

この時、$I(\Delta)$の2次変分$\langle I(\Delta)\rangle_t$を以下の様に定める。
\begin{equation}
    \langle I(\Delta) \rangle_t := A_t
\end{equation}
このことより、$I(\Delta)$の2次変分が、a.s.に同一なパスを持つという意味において、一意に存在するということが言える。

この2次変分が教科書の定義と合致することは、Karatzas, and Shreveの以下の定理より分かる。
\begin{thm}[]
    Let $X \in \mathcal{M}_{2}^c$.
    For partitions $\Pi$ of $[0, t]$, we have
    \begin{equation}
        \plim_{||\Pi|| \to 0} V_{t}^{2} (\Pi) = \langle X \rangle_t
    \end{equation}
\end{thm}


\bibliography{Hitotsubashi2023.bib}
\bibliographystyle{econ-aea.bst}

\end{document}