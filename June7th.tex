\documentclass{ltjsarticle}


% 数式
\usepackage[]{amsmath, amssymb, amsfonts, amsthm}
\usepackage{physics}
\usepackage{enumerate}

% 数式以外
\usepackage[]{comment}
\usepackage[longnamesfirst]{natbib}
\usepackage[]{graphicx}
\usepackage[]{here}
\numberwithin{equation}{subsection}


% 定理や定義の書式

% 定義
\theoremstyle{definition}
\newtheorem{dfn}{Definition}[subsection]
% 定理
\newtheorem{thm}[dfn]{Theorem}
\newtheorem{thm*}[dfn]{Theorem*}
% 命題
\newtheorem{prop}[dfn]{Proposition}
% 補題
\newtheorem{lem}[dfn]{Lemma}
% 注意
\newtheorem{rem}[dfn]{Rem}
% アルゴリズム
\newtheorem{algo}[dfn]{Algorithm}
% 例
\newtheorem{ex}[dfn]{Example}


\begin{document}

\title{確率解析勉強会第1回}
\author{中津陽}
\maketitle

\begin{abstract}
    この資料は、「」著「」を元に、確率解析の学習を行うために作成された。
\end{abstract}

\section{一般的な確率論}
\subsection{無限確率空間}
 なぜ$\sigma$-加法族を考える必要があるか。 

\begin{dfn}[$\sigma$-加法族]
    \label{dfn:sigma-alg}
    $\Omega \neq \phi$, $\mathcal{F} \subset 2^{\Omega}$とする。
    $\mathcal{F}$が$\sigma$加法族であるとは、次を満たすことである。
    \begin{enumerate}[(i)]
        \item $\phi \in \mathcal{F}$
        \item $A \in \mathcal{F} \Rightarrow A^{c} \in \mathcal{F}$ 
        \item 
            $
            \forall n \in \mathbb{N} \qty[ A_n \in \mathcal{F} ] \Rightarrow \bigcup_{n=1}^{\infty} A_n \in \mathcal{F}
            $
    \end{enumerate}
\end{dfn}

Definition\ref{dfn:sigma-alg}(i), (iii)より、$A \in \mathcal{F} \land B \in \mathcal{F} \Rightarrow A \cup B \in \mathcal{F}$が成立する。
\begin{proof}
    集合列$A, B, \phi, \phi, \dots$を考える。
    ただし、$A, B \in \mathcal{F}$である。
    Definition\ref{dfn:sigma-alg}(i)より、$\phi \in \mathcal{F}$である。
    したがって、Definition\ref{dfn:sigma-alg}(iii)より、
    $A \cup B = A \cup B \cup \phi \cup \phi \dots \in \mathcal{F}$
    が言える。
\end{proof}

同様にして、ある自然数$N$に対し、$\forall n \in \{1,2,\dots,N \} \qty[A_n \in \mathcal{F}] \Rightarrow \bigcup_{n=1}^{N} A_n \in \mathcal{F}$が成立する。
\begin{proof}
    $N=k$において、題意が満たされるものと仮定する。
    したがって、$\bigcup_{n=1}^{k}A_n \in \mathcal{F}$である。
    $A_{k+1} \in \mathcal{F}$をとる。
    % この時、$\forall n \in \{1,2,\dots,k,k+1 \} \qty[A_n \in \mathcal{F}]$は真である。
    この時、上述の証明より$\bigcup_{n=1}^{k+1}A_n = \bigcup_{n=1}^{k}A_n \cup A_{k+1} \in \mathcal{F}$
    $N=1$の場合は自明なので、以上から再起的に題意は示された。
\end{proof}

Definition\ref{dfn:sigma-alg}(ii),(iii)より、
$\forall n \in \mathbb{N} [A_n \in \mathcal{F}] \Rightarrow \bigcap_{n=1}^{\infty} A_n \in \mathcal{F}$が示される。
\begin{proof}
    $\bigcap_{n=1}^{\infty} A_n = \qty( \bigcup_{n=1}^{\infty} A_{n}^{c} )^{c}$より、Definition\ref{dfn:sigma-alg}(ii),(iii)から、$\bigcap_{n=1}^{\infty} A_n = \qty( \bigcup_{n=1}^{\infty} A_{n}^{c} )^{c} \in \mathcal{F}$である。
\end{proof}

同様にして、ある自然数$N$に対し、$\forall n \in \{1,2,\dots,N \} \qty[A_n \in \mathcal{F}] \Rightarrow \bigcap_{n=1}^{N} A_n \in \mathcal{F}$が成立する。
\begin{proof}
    略
\end{proof}

$\Omega = \phi^{c}$なので、$\Omega \in \mathcal{F}$が言える。

\begin{dfn}[確率測度]\mbox{}
    \label{dfn:measure}
    $\Omega \neq \phi$を集合、 $\mathcal{F} \subset 2^{\Omega}$を$\sigma$-加法族とする。以下を満たすとき、関数$\mathrm{P}:\mathcal{F} \in A \to x \in [0,1]$を確率測度という。
    \begin{enumerate}[(i)]
        \item $\mathrm{P}(\Omega)=1$
        \item $\qty{A_n}$が$\mathcal{F}$に属する互いに素な集合列ならば、($i.e. \: \forall n,m \in \mathbb{N}(n \neq m), A_n \cap A_m = \phi$)
        次が成立する。
        \begin{equation}
            \mathrm{P} \qty(\bigcap_{n=1}^{\infty} A_n) = \sum_{n=1}^{\infty} \mathrm{P} \qty(A_n)
        \end{equation}
    \end{enumerate}
    この時、$\mathrm{P}(A) \in [0,1]$の値を$A$の確率と呼ぶ。また、(ii)の性質のことを可算加法性という。
\end{dfn}

\begin{ex}[]
    $\qty[0,1]$から、無作為に実数を選ぶことに対する数学モデルを、その確率が区間上に一様に分布するよう考える。
    開区間$\qty[a,b]$の確率は以下のように定義される。
    \begin{equation}
        \label{ex:01lebesgue}
        \mathrm{P} \qty[a,b] = b-a , \: 0 \leq a \leq b \leq 1
    \end{equation}
    $\qty[0,1]$上のこの確率測度はLebesgue測度と呼ばれ、$\mathcal{L}$とも記される。
    $\mathrm{P}[a,a]=0$より、区間上の1点に対する確率は0であるとわかる。
    従って、$\mathrm{P}(a,b)=\mathrm{P}[a,b]$であると言える。

    \ref{ex:01lebesgue}と確率測度の性質より、確率が決まる集合の全体を考えると、
    それは閉区間の全体を含む最小の$\sigma$-加法族であるとわかる。
    開区間は閉区間列の和集合として表されるため、この$\sigma$-加法族は全ての開区間を含む。

    このような$\sigma$-加法族を$\qty[0,1]$の部分集合のBorel$\sigma$-加法族といい、$\mathcal{B}\qty[0,1]$
    と表す。このBorel$\sigma$-加法族に属する集合は、Borel集合と呼ばれる。
    これらの事象の確率は、閉区間の確率を決めると自動的に定まることが言える。
\end{ex}

\begin{ex}[]
    コインを無限回投げるとし、$\Omega_{\infty}$を起こり得る全ての結果の集合とする。
    各コイン投げでの表の確率は$p > 0$、裏の確率は$q = 1-p > 0$であるものとし、
    各回のコイン投げは全て独立であるものとする。
    この確率的試行に対応する確率測度を構成する。
    そのためには、先ほどの例題から類推して、有限回のコイン投げで表すことのできる全ての集合を集めた集合族を含む最小の
    $\sigma$-加法族を構成し、有限回のコイン投げにより表される集合の確率から、その$\sigma$-加法族に属する
    全ての集合の確率が定まることを言えば良い。

    そのようにして構成した$\sigma$-加法族の要素の中には、確率の値の計算が困難なものもある。
    この集合族を$\mathcal{F}_{\infty}$と定義した上で、次のような集合$A$を考える。
    \begin{equation}
        \label{ex:114-A}
        A = \qty{ \omega \in \Omega | \lim_{n \to \infty} \frac{H_n(\omega_1 \dots \omega_n)}{n} = \frac{1}{2} }
    \end{equation}
    ただし、$H_n(\omega_1 \dots \omega_n)$は最初の$n$回のコイン投げの$H$の回数である。
    つまり、$A$は表が出る比率の長期的平均が$\frac{1}{2}$であるような、表・裏の列全体からなる集合であると言える。
    $A \in \mathcal{F}_{\infty}$を示す。そのために、$m,n \in \mathbb{N}_{>0}$に対し、次の集合を定義する。
    \begin{equation}
        \label{ex:114-S}
        S_{n,m} = \qty{ \omega \in \Omega ; \abs{\frac{H_n(\omega_1 \dots \omega_n)}{n} - \frac{1}{2}} \leq \frac{1}{m} }
    \end{equation}
    $S_{n,m}$は、最初の$n$回のコイン投げ結果を用いて表される全ての事象からなる$\sigma$-加法族$\mathcal{F}_n$に
    属しており、その確率は$n,m$が判明次第、定義できる。

    コイン投げの列$\omega=\omega_1 \omega_2 \cdots$が$\omega \in A$を満たす必要十分条件は、
    $\forall m \in \mathbb{N} \qty[ \exists N \in \mathbb{N} \qty[ \forall n \geq N \qty[\omega \in S_{n,m}] ] ]$
    が成立することであり、この命題は次のように言い換えられる。
    \begin{eqnarray*}
        && \forall m \in \mathbb{N} \qty[ \exists N \in \mathbb{N} \qty[ \forall n \geq N \qty[\omega \in S_{n,m}] ] ] \\ 
        &=& \forall m \in \mathbb{N} \qty[ \exists N \in \mathbb{N} \qty[ \omega \in \bigcap_{n=N}^{\infty} S_{n,m} ] ] \\ 
        &=& \forall m \in \mathbb{N} \qty[ \omega \in \bigcup_{N=1}^{\infty} \bigcap_{n=N}^{\infty} S_{n,m} ] \\ 
        &=& \omega \in \bigcap_{m=1}^{\infty} \bigcup_{N=1}^{\infty} \bigcap_{n=N}^{\infty} S_{n,m} 
    \end{eqnarray*}
    従って、$A = \bigcap_{m=1}^{\infty} \bigcup_{N=1}^{\infty} \bigcap_{n=N}^{\infty} S_{n,m}$である。
    従って、$A \in \mathcal{F}_{\infty}$が言える。
    測度が$\mathcal{F}_{\infty}$上で定義されているので、確率$\mathrm{P}(A)$の値は定まると言える。
    結論的には、大数の強法則より、$p=\frac{1}{2}$ならば$\mathrm{P}(A)=1$、$p \neq \frac{1}{2}$ならば$\mathrm{P}(A)=0$であると言える。
\end{ex}

\begin{dfn}[ほとんど確実に]
    $(\Omega , \mathcal{F} , \mathrm{P})$を確率空間とする。
    $A \in \mathcal{F}$が$\mathrm{P}(A)=1$の時、ほとんど確実(almost surely)に事象$A$が起こるという。
\end{dfn}


ルベーグ下積和を次で定義する。
\begin{equation}
    LS_{\Pi}^{-}(X) = \sum_{k=1}^{\infty} y_k \mathrm{P} (A_k)
\end{equation}
分割点間の距離の最大値$||\Pi||$が0に近づくにつれて、$\qty[-\infty , \infty]$の範囲でこの下積和は収束し、
この極限値をルベーグ積分$\int_{\Omega} X(\omega) d \mathrm{P}(\omega)$として定義する。


$\forall \omega \in \Omega \qty[0 \leq X(\omega) < \infty]$を仮定していた。
この条件に反する$\omega$の集合が測度0ならば、$\infty * 0 = 0 * \infty = 0$なので、
この積分には影響を及ぼさない。
一方、$\mathrm{P} \qty{\omega ; X(\omega) \geq 0} = 1$かつ$\mathrm{P} \qty{\omega ; X(\omega) = \infty} >0$
ならば、$\int_{\Omega} X(\omega) d \mathrm{P}(\omega)= \infty$と定義する。


確率変数$X$が、正負両方の値を取りうる場合を考える。
確率変数$X$の正の部と負の部を次で定義する。
\begin{equation}
    X^+ \qty(\omega) = max \qty{X(\omega), 0}, \; X^- (\omega) = max \qty{ -X(\omega), 0}
\end{equation}
これらはともに非負の確率変数であり、$X=X^+ - X^-, \; |X| = X^+ + X^-$が成立する。
$\int_{\Omega} X^+(\omega) d \mathrm{P}(\omega)$, $\int_{\Omega} X^-(\omega) d \mathrm{P}(\omega)$
はどちらも上記の手法により定義できる。
また、少なくとも一方が$\infty$でない場合、$\qty[-\infty , \infty]$の範囲で、次を定義できる。
まだ、積分の線形性は示されていないので、これが定義である。
\begin{equation}
    \label{dfn:L-integral}
    \int_{\Omega} X(\omega) d \mathrm{P}(\omega) = \int_{\Omega} X^+ (\omega) d \mathrm{P}(\omega) - \int_{\Omega} X^-(\omega) d \mathrm{P} (\omega)
\end{equation}
次が満たされる時、$X$は可積分であるという。
\begin{equation}
    \int_{\Omega} X^+ (\omega) d \mathrm{P}(\omega) < \infty, \int_{\Omega} X^-(\omega) d \mathrm{P} (\omega) < \infty
\end{equation}
また、$\int_{\Omega} X^+(\omega) d \mathrm{P}(\omega) = \infty$かつ$\int_{\Omega} X^-(\omega) d \mathrm{P}(\omega) = \infty$の場合は、$\infty - \infty$となり、$\int_{\Omega} X(\omega) d \mathrm{P}(\omega)$は不定である(定義されない)。


ルベーグ積分の基本性質は次の通りである。
\begin{thm}[]
    Xを確率空間$\qty(\Omega , \mathcal{F}, \mathrm{P})$上の確率変数とする。
    \begin{enumerate}[(i)]
        \item Xがa.s.に取りうるのは、有限個の値$y_1, y_2, y_3, \dots , y_n$だけならば、次が成立する。
        \begin{equation}
            \int_{\Omega} X(\omega) dP(\omega) = \sum_{k=0}^{n} y_k \mathrm{P} \qty{X=y_k}
        \end{equation}
        \item 確率変数$X$が可積分であるための必要十分条件は、以下である。
        \begin{equation}
            \int_{\Omega} |X(\omega)| d \mathrm{P}(\omega) < \infty
        \end{equation}
        ただし、$Y$は$(\Omega , \mathcal{F} , \mathrm{P})$上の別の確率変数とする。
        この性質を可積分性という。
        \item $X \leq Y a.s.$かつ、$\int_{\Omega} X(\omega ) d \mathrm{P} (\omega )$と
        $\int_{\Omega} Y(\omega ) d \mathrm{P} (\omega )$が定義されているならば、
        次が成立する。
        \begin{equation}
            \int_{\Omega} X(\omega ) d \mathrm{P} (\omega ) \leq \int_{\Omega} Y(\omega ) d \mathrm{P} (\omega )
        \end{equation}
        この性質のことを大小関係の保存という。
        特に、$X=Y a.s.$で、どちらかの積分が定義されているならば、両方が定義され次が成立する。
        \begin{equation}
            \int_{\Omega} X(\omega ) d \mathrm{P} (\omega ) = \int_{\Omega} Y(\omega ) d \mathrm{P} (\omega )
        \end{equation}
        \item $\alpha ,\beta \in \mathbb{R}$かつ$X,Y$が可積分、または$\alpha ,\beta \in \mathbb{R}_{>0}$かつ$X , Y$が非負ならば、次が成立する。
        \begin{equation}
            \int_{\Omega} (\alpha X(\omega) + \beta Y(\omega)) d \mathrm{P} (\omega ) = \alpha \int_{\Omega} X(\omega) d \mathrm{P} (\omega) + \beta \int_{\Omega} (\omega) d\mathrm{P} (\omega)
        \end{equation}
        この性質を線型性という。
    \end{enumerate}
\end{thm}

\begin{proof}
    (i) $X \geq 0 a.s$の場合を考える。
        $\qty{y_k}$の中に$0$が含まれていないなら、$y_0=0$を加える。
        これは、$X$が一般の確率変数である場合に、$X^+ - X^-$を考える必要があるからだと思われる。

        必要ならば、添え字を振り直して$0=y_0 < y_1 < y_2 < \cdots < y_n$と仮定する。
        これらを分割点として用いて、$A_k = \qty{y_k \leq X < y_{k+1}} = \qty{X=y_k}$となり、
        ルベーグ下積和は以下のように書ける。
        \begin{equation}
            LS_{\Pi}^{-}(X) = \sum_{k=1}^{n} y_k \mathrm{P} (A_k)
        \end{equation}
        分割点を増やしてもルベーグ下積和は変化しないので、これはルベーグ積分の説明に整合的である。


    (ii)後ほど、(iii),(iv)を用いて示す。


    (iii)$X \leq Y a.s.$ならば、$X^+ \leq Y^+$かつ$X^- \geq Y^-$a.s.である。
    $X^+ \leq Y^+ a.s.$なので、任意の分割$\Pi$に対し、ルベーグ下積和は$LS_{\Pi}^{-}(X^+) \leq LS_{\Pi}^{-}(Y^+)$を満たす。従って、次が成立する。
    \begin{equation}
        \label{eq:Li1}
        \int_{\Omega} X^+(\omega) d\mathrm{P} (\omega) \leq \int_{\Omega} Y^+(\omega) d\mathrm{P} (\omega)
    \end{equation}
    ほとんど確実に$X^- \geq Y^-$なので、次も成立する。
    \begin{equation}
        \label{eq:Li2}
        \int_{\Omega} X^-(\omega) d\mathrm{P} (\omega) \geq \int_{\Omega} Y^-(\omega) d\mathrm{P} (\omega)
    \end{equation}
    (\ref{dfn:L-integral}),(\ref{eq:Li1}),(\ref{eq:Li2})より、示される。


    (iv) の証明は、ルベーグ積分の正確な分析が必要となるので、今回はパスする。


    (ii) (iii)と(iv)を用いて示す。必要性は(iv)より自明なので、十分性を示す。$\forall \omega \in \Omega$に対し、$|X|=X^+ + X^-$かつ$X^+, X^- \geq 0$なので、$\forall \omega \in \Omega$に対し、$X^+ \leq |X|$かつ$X^- \leq |X|$である。もし$\int_{\Omega} |X(\omega)| d\mathrm{P} (\omega) < \infty$ならば、(iii)より$\int_{\Omega} X^+(\omega) d\mathrm{P} (\omega) < \infty$かつ$\int_{\Omega} X^-(\omega) d\mathrm{P} (\omega) < \infty$が真であると言え、$X$は定義より可積分であると言える。
\end{proof}

\begin{rem}[]
    確率変数を$A \in \mathcal{F}$で積分する。$A \in \mathcal{F}$に対し、次を定義する。
    \begin{equation}
        \int_{A} X(\omega) d\mathrm{P} (\omega) = \int_{\Omega} \mathbb{I}_A(\omega) d\mathrm{P} (\omega)
    \end{equation}
    ただし$\mathbb{I}_A(\omega)$は次式の定義函数(確率変数)である。
    \begin{equation}
        \mathbb{I}_A(\omega) =
        \begin{cases}
            1 & \omega \in A \\
            0 & \omega \notin A 
        \end{cases}
    \end{equation}
    $A \cap B = \phi$ならば、$\mathbb{I}_A + \mathbb{I}_B = \mathbb{I}_{A\cup B}$であり、
    線形性より次が得られる。
    \begin{equation}
        \int_{A\cup B} X(\omega) d\mathrm{P}(\omega) = \int_{A} X(\omega) d\mathrm{P} (\omega) + \int_{B} X(\omega) d\mathrm{P} (\omega)
    \end{equation}
\end{rem}


\begin{dfn}[期待値]
    $X$を確率空間$(\Omega , \mathcal{F} , \mathrm{P})$上の確率変数とする。
    $X$の期待値(expectation, or expected value)は次で定義される。
    \begin{equation}
        \mathbb{E}X = \int_{\Omega} X(\omega) d\mathbb{P} (\omega)
    \end{equation}
\end{dfn}
この定義が意味を持つのは、右辺の積分が定義できる場合である。
$\mathbb{E}X^+$, $\mathbb{E}X^-$の両方が有限であるとき、$\mathbb{E}X < \infty$であり、$X$は可積分であるという。


ルベーグ積分の基本的性質を、期待値の言葉に置き換えて述べ直す。
\begin{thm}[]
    $X,Y$を確率空間$(\Omega , \mathcal{F} , \mathrm{P})$上の異なる確率変数とする。


    (i) $X$の取り得る値が可算個($x_0, x_1, \dots, x_n, \dots$)ならば、次が成立する。
    \begin{equation}
        \mathbb{E}X = \sum_{\omega \in \Omega} X(\omega) \mathrm{P} \qty{\omega}
    \end{equation}
    ただし、$\Omega = \qty{x_0, x_1, \dots, x_n, \dots}$とする。


    (ii) 確率変数$X$が可積分であるための必要十分条件は、
    \begin{equation}
        \mathbb{E} |X| < \infty
    \end{equation}
    である。


    (iii) $X \leq Y a.s.$であるとする。$X, Y$がそれぞれ可積分、または$X, Y \geq 0 a.s.$ならば、
    次が成立する。
    \begin{equation}
        \mathbb{E} X = \mathbb{E} Y
    \end{equation}
\end{thm}


% \bibliography{Hitotsubashi2023.bib}
% \bibliographystyle{econ-aea.bst}

\end{document}